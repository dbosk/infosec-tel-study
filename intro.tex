\section{Introduction}

We have a course on security.
As with any introductory course on this topic, this course has a module on 
authentication.
How can we teach proper authentication to our students and with some retention?

Identity-based authentication with password verification has been the dominant 
form of authentication throughout the history of computing.
Any person who has signed up for accounts on the Web, will surely have 
encountered the use of an email address as identifier and a matching password 
for authentication.
This surely applies to students who take a technical programme at university 
level.
The password should, for \enquote{security reasons}, consist of at least eight 
characters, of which one must be an upper case, one a lower case, one a digit 
and one a special character --- or some variation of such requirements.

\subsection{What's the problem and why?}
% Why is it a problem?
% Why is it important, why care?

The current practice of identity-based authentication with complex passwords 
has been shown by decades of research to be bad practice.
However, since it is the practice and our students experience it daily, we 
argue that this makes the bad practice heavily reinforced and that it continues 
to be reinforced every day.

In the authentication module we cover research that shows that the 
\enquote{complex passwords} policy yields passwords that are easy to crack.
We also cover decades old research of alternatives to identity-with-password 
based authentication, alternatives which are better from security, usability 
and privacy standpoint.

However, we have observed that (some) students have a tendency to still act 
according to their old behaviour even after the module.
\Ie upon the question \enquote{how to do authentication?} they reply 
\enquote{email as username and a complex password}.
We have also observed an it-doesn't-happen-to-me attitude and, particularly, 
that this attitude change when something actually happens to that \enquote{me}.

This has left us with the following questions.
\begin{rqs}
\item\label{rq-nochange}
  Why do some students retain their old behaviour despite the course module?

\item\label{rq-unlearn}
  How can we make them unlearn something very reinforced and relearn the 
  correct?

\item\label{rq-retention}
  How can we improve how long do the students retain what they have 
  (re)learned?
\end{rqs}

