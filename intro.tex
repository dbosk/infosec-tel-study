\section{Introduction}

I have given a course on security once per year for eight years.
As with any introductory course on this topic, this course has a module on 
authentication.
How can we teach proper authentication to our students?

Identity-based authentication with password verification has been the dominant 
form of authentication throughout the history of computing.
Any person who has signed up for accounts on the Web, will surely have 
encountered the use of an email address as identifier and a matching password 
for authentication.
This surely applies to students who take a technical programme at university 
level.
The password should, for \enquote{security reasons}, consist of at least eight 
characters, of which one must be an upper case, one a lower case, one a digit 
and one a special character --- or some variation of such requirements.

\subsection{What's the problem and why?}
% Why is it a problem? Research gap left by other approaches?
% Why is it important, why care?

We argue that this is reinforced every day.

\dots
RQ1: How to unlearn something very reinforced and relearn the correct? (From 
identity-based authentication with passwords to attributes and crypto.)

RQ2: How long do the students retain this?


\subsection{What's the approach?}
% What's the approach? How to solve the problem?

\dots

\subsection{What are the findings?}
% What's the findings? How was it evaluated, what are the results, limitations, 
% what remains to be done?

\dots

