\section{What's the approach?}
% What's the approach? How to solve the problem?

We take \iac{DBR} approach to investigate the research questions.

\paragraph{Learning analytics}

The students work in groups to prepare for seminars, do lab work \etc.
We record these group memberships: the students sign up for a group in the 
\ac{VLE}.
However, we cannot track how students communicate during preparations, since 
they do that in the real world during their own time.
But if we know the group memberships, we know at least who has 
\emph{potentially} communicated with whom.

During seminars, there are also group discussions, but these groups are 
randomly assigned.
These group memberships must also be recorded: we record them manually from the 
seminar session.

We also want to record when (or if) students access the resources in the 
course:
videos,
instructions,
papers,
\etc.
(The \ac{VLE} records this.)
This way we can see if students still exhibiting the old behaviour did not take 
part of the prescribed material and simply didn't study.
However, there is still the possibility that they take part of the material 
second-hand through the group.
\Eg one person in the group downloads the paper and prints or transmits a copy 
to the others.
(We have made such observations in the past.)
Then we have the group memberships from above to see if such an explanation is 
possible for any individual student.

\paragraph{A priori and a posteriori knowledge and motivation}

\dots
