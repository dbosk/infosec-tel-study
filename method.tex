\section{What's the approach?}
% What's the approach? How to solve the problem?

We take \iac{DBR} approach to investigate the research questions.
\Cref{rq-nochange} focuses on an initial problem that is expected to give some 
guidance to the answers of \cref{rq-unlearn,rq-retention}.
\Cref{rq-unlearn,rq-retention} require iterations, particularly to improve the 
retention of \cref{rq-retention}.

\paragraph{Learning analytics}

We will record when (or if) students access the resources in the course:
videos,
instructions,
papers,
\etc.
The \ac{VLE} records this, some parts are recorded by ScalableLearning and 
YouTube.
This way we can see if the students still exhibiting the old behaviour did not 
take part of the prescribed material and simply didn't study.

However, there is still the possibility that they take part of the material 
second-hand through friends.
The students work in groups to prepare for seminars, do lab work \etc.
We record these group memberships: the students sign up for a group in the 
\ac{VLE}.
However, we cannot track how students communicate during preparations, since 
they do that in the real world during their own time.
But if we know the group memberships, we know at least who has 
\emph{potentially} communicated with whom.
We have previously observed that if they study together, one person in the 
group downloads the paper and prints or transmits a copy to the others.
But if we have the group memberships from above to see if such an explanation 
is possible for any individual student.

During seminars, there are also group discussions, but these groups are 
randomly assigned.
These group memberships must also be recorded: we record them manually from the 
seminar session.

We note again, this recorded data can only represent possibilities, not 
certainties.

\paragraph{A priori and a posteriori knowledge and motivation}

We also want to evaluate the change in the students that the module induces.
For this, we must evaluate the students' knowledge and motivation a priori to 
the module and a posteriori.

In the first round we will do this by semi-structured interviews, refining 
towards structured interviews for each iteration.
Before the module starts, we will conduct interviews with a random sample of 
students.
During this interview we want to evaluate the student's
\begin{itemize}
  \item attitude towards and emotional engagement with the subject,
  \item knowledge and beliefs related to the subject.
\end{itemize}

After the module, we will conduct the same interview again, to evaluate the 
possible change.

\paragraph{Ground truth}

Since we cannot tell from the \ac{LA} data whether they didn't study or they 
accessed the material through their peers, we must inquire about this too.
We will do this by interviewing select individuals exhibiting the behaviour.
However, these interviews will take place after the course has ended and their 
grades are set, to decrease any fear of affecting their grade.
