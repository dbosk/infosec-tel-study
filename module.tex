\hypertarget{non-interactive-learning-sessions}{%
\subsection{Non-interactive learning
sessions}\label{non-interactive-learning-sessions}}

\href{https://youtu.be/aqYyxhGeRQc}{Introduction to Authentication}
introduces the basic ideas of authentication, the role of trust and
privacy, and some important principles.

\href{https://youtu.be/OPVZQ5TJnqk}{Bootstrapping Authentication}
introduces the problem of bootstrapping and its relation to recovery in
case of failure and gives some examples of popular solutions in
practice.

\href{https://youtu.be/wEc0FFxw6aY}{User-to-Machine Authentication}
introduces the topic of user authentication from the perspective of a
machine, how can a machine authenticate its user?

\href{https://youtu.be/A9-nBFnZszk}{Something-You-Know} discusses
various aspects of the security of performing authentication with
something-you-know.

\href{https://youtu.be/W_7DJ5T4TDs}{Something-You-Have} gives an
overview of the idea of this approach to user authentication. In
essence, we can embed cryptographic keys in hardware that will then be
difficult to clone (copy), unlike the passwords in the
something-you-know approach.

\href{https://youtu.be/71GlqDmesRc}{Machine-to-User Authentication} is
the opposite problem of user-to-machine authentication: how can a user
tell something legitimate from something illegitimate?

\hypertarget{reading}{%
\subsection{Reading}\label{reading}}

Why we want to do this and how we can accomplish this is treated in
Chapter 4 of {(Gollmann 2011)}. Anderson also treats this
topic~{(Anderson 2008 Chap.~2)}, although in a wider perspective with
less technical details.

For the treatment of anonymous credentials, we refer to {(Camenisch,
Lehmann, and Neven 2012)} and ~{(Lee et al. 2014)}.

\hypertarget{references}{%
\paragraph{References}\label{references}}
\addcontentsline{toc}{paragraph}{References}

\hypertarget{refs}{}
\leavevmode\hypertarget{ref-Anderson2008sea}{}%
Anderson, Ross J. 2008. \emph{Security Engineering: A Guide to Building
Dependable Distributed Systems}. 2nd ed. Indianapolis, IN: Wiley.
\url{http://www.cl.cam.ac.uk/~rja14/book.html}.

\leavevmode\hypertarget{ref-AnonymousCredentials}{}%
Camenisch, J., A. Lehmann, and G. Neven. 2012. ``Electronic Identities
Need Private Credentials.'' \emph{IEEE Security Privacy} 10 (1): 80--83.
\url{https://doi.org/10.1109/MSP.2012.7}.

\leavevmode\hypertarget{ref-Gollmann2011cs}{}%
Gollmann, Dieter. 2011. \emph{Computer Security}. 3rd ed. Chichester,
West Sussex, U.K.: Wiley.

\leavevmode\hypertarget{ref-AnonPass-magazine}{}%
Lee, M. Z., A. M. Dunn, J. Katz, B. Waters, and E. Witchel. 2014.
``Anon-Pass: Practical Anonymous Subscriptions.'' \emph{IEEE Security
Privacy} 12 (3): 20--27. \url{https://doi.org/10.1109/MSP.2013.158}.

\hypertarget{interactive-learning-session}{%
\subsection{Interactive learning
session}\label{interactive-learning-session}}

There is one interactive session for this module, see the
\href{https://portal.miun.se/web/student/schedule}{schedule} for date
and time. Watch the videos linked above before the session. During the
session we will
\href{https://ver.miun.se/courses/security/infosakc/auth-session.pdf}{summarize
(together) the most important parts, discuss the most
difficult/ambiguous/strange/counter-intuitive parts}.

\hypertarget{seminarworkshop-sessions}{%
\subsection{Seminar/workshop sessions}\label{seminarworkshop-sessions}}

\href{https://ver.miun.se/courses/security/infosakc/pwdeval.pdf}{Evaluating
and designing authentication} is a lab/seminar which focuses on
evaluating the usability and security of common authentication methods.
It also looks towards better alternatives. This is spread over two
sessions. (See the
\href{https://portal.miun.se/web/student/schedule}{schedule} for date
and time.)
