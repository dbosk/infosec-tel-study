\title{%
  How to unlearn and relearn authentication
}
\author{Daniel Bosk}
\affiliation{%
  \institution{KTH EECS}
}

\setcopyright{none}

\begin{abstract}
  % What's the problem?
% Why is it a problem? Research gap left by other approaches?
% Why is it important? Why care?
% What's the approach? How to solve the problem?
% What's the findings? How was it evaluated, what are the results, limitations, 
% what remains to be done?

\paragraph{The problem}

We have a module on authentication for an introductory course on security.
During the learning activities we particularly cover the badness of requiring 
complex passwords.
Yet, on the exam we still find students who say that complex passwords is a 
good idea.
This has prompted us to ask the questions:
\begin{rqs}
\item%\label{rq-nochange}
  Why do some students retain their old behaviour despite the course module?

\item%\label{rq-unlearn}
  How can we make them unlearn something very reinforced and relearn the 
  correct?

\item%\label{rq-retention}
  How can we improve how long do the students retain what they have 
  (re)learned?
\end{rqs}

\paragraph{Who we support}

We primarily want to support the teacher, to evaluate and improve the learning 
design, thus indirectly supporting the students.
\Eg the inquiry might lead to a finding saying we need better feedback for 
students, to make them aware they haven't learned what they should have.

\paragraph{What learning activities we address}

The module is part of an online course.
The module consists of a flipped classroom session (six videos, one interactive 
session online) and a workshop assignment (two online seminars interleaved with 
paper reading and laboratory work).

\paragraph{Data we need and how to collect them}

We will use \ac{LA} to record how the students access the resources and the 
groups they work in.
This can be recorded with the tools used: \ac{VLE}, video meeting tool.
We use the assessment of the course to record how the students perform.
We will also conduct interviews to study the more qualitative aspects of the 
students' behaviour and reasons for the outcomes.

\paragraph{Motivation for the analysis}

We must analyse the \ac{LA} data to distinguish patterns that can guide our 
investigation.
The nature of \cref{rq-nochange} require qualitative data to explain the 
patterns that we might see.
That need is provided by the interviews.
The interviews can also guide us for \cref{rq-unlearn,rq-retention}.
The assessment of the course will tell us whether our learning design for 
\cref{rq-unlearn} succeeded or not.
Finally, to assess retention (\cref{rq-retention}), we must follow up with the 
students.


\end{abstract}

\maketitle



\section{Introduction}

I have given a course on security once per year for eight years.
As with any introductory course on this topic, this course has a module on 
authentication.
How can we teach proper authentication to our students?

Identity-based authentication with password verification has been the dominant 
form of authentication throughout the history of computing.
Any person who has signed up for accounts on the Web, will surely have 
encountered the use of an email address as identifier and a matching password 
for authentication.
This surely applies to students who take a technical programme at university 
level.
The password should, for \enquote{security reasons}, consist of at least eight 
characters, of which one must be an upper case, one a lower case, one a digit 
and one a special character --- or some variation of such requirements.

\subsection{What's the problem and why?}
% Why is it a problem? Research gap left by other approaches?
% Why is it important, why care?

We argue that this is reinforced every day.

\dots
RQ1: How to unlearn something very reinforced and relearn the correct? (From 
identity-based authentication with passwords to attributes and crypto.)

RQ2: How long do the students retain this?


\subsection{What's the approach?}
% What's the approach? How to solve the problem?

\dots

\subsection{What are the findings?}
% What's the findings? How was it evaluated, what are the results, limitations, 
% what remains to be done?

\dots


%%% REFERENCES %%%

\printbibliography
